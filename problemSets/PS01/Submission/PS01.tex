\documentclass[12pt,letterpaper]{article}
\usepackage{graphicx,textcomp}
\usepackage{natbib}
\usepackage{setspace}
\usepackage{fullpage}
\usepackage{color}
\usepackage[reqno]{amsmath}
\usepackage{amsthm}
\usepackage{fancyvrb}
\usepackage{amssymb,enumerate}
\usepackage[all]{xy}
\usepackage{endnotes}
\usepackage{lscape}
\newtheorem{com}{Comment}
\usepackage{float}
\usepackage{hyperref}
\newtheorem{lem} {Lemma}
\newtheorem{prop}{Proposition}
\newtheorem{thm}{Theorem}
\newtheorem{defn}{Definition}
\newtheorem{cor}{Corollary}
\newtheorem{obs}{Observation}
\usepackage[compact]{titlesec}
\usepackage{dcolumn}
\usepackage{tikz}
\usetikzlibrary{arrows}
\usepackage{multirow}
\usepackage{xcolor}
\newcolumntype{.}{D{.}{.}{-1}}
\newcolumntype{d}[1]{D{.}{.}{#1}}
\definecolor{light-gray}{gray}{0.65}
\usepackage{url}
\usepackage{listings}
\usepackage{color}

\definecolor{codegreen}{rgb}{0,0.6,0}
\definecolor{codegray}{rgb}{0.5,0.5,0.5}
\definecolor{codepurple}{rgb}{0.58,0,0.82}
\definecolor{backcolour}{rgb}{0.95,0.95,0.92}

\lstdefinestyle{mystyle}{
	backgroundcolor=\color{backcolour},   
	commentstyle=\color{codegreen},
	keywordstyle=\color{magenta},
	numberstyle=\tiny\color{codegray},
	stringstyle=\color{codepurple},
	basicstyle=\footnotesize,
	breakatwhitespace=false,         
	breaklines=true,                 
	captionpos=b,                    
	keepspaces=true,                 
	numbers=left,                    
	numbersep=5pt,                  
	showspaces=false,                
	showstringspaces=false,
	showtabs=false,                  
	tabsize=2
}
\lstset{style=mystyle}
\newcommand{\Sref}[1]{Section~\ref{#1}}
\newtheorem{hyp}{Hypothesis}

\title{Problem Set 1}
\date{Due: October 1, 2021}
\author{Applied Stats/Quant Methods 1}

\begin{document}
	\maketitle
	
	\section*{Instructions}
	\begin{itemize}
		\item Please show your work! You may lose points by simply writing in the answer. If the problem requires you to execute commands in \texttt{R}, please include the code you used to get your answers. Please also include the \texttt{.R} file that contains your code. If you are not sure if work needs to be shown for a particular problem, please ask.
		\item Your homework should be submitted electronically on GitHub in \texttt{.pdf} form.
		\item This problem set is due before 8:00 on Friday October 1, 2021. No late assignments will be accepted.
		\item Total available points for this homework is 100.
	\end{itemize}
	
	\vspace{1cm}
	\section*{Question 1 (50 points): Education}

A school counselor was curious about the average of IQ of the students in her school and took a random sample of 25 students' IQ scores. The following is the data set:\\
\vspace{.5cm}

\lstinputlisting[language=R, firstline=40, lastline=40]{PS01.R}  

\vspace{1cm}

\begin{enumerate}
	\item Find a 90\% confidence interval for the average student IQ in the school.\\
	
	Question1 - Part 1
	y <- c(105, 69, 86, 100, 82, 111, 104, 110, 87, 108, 87, 90, 94, 113, 112, 98, 80, 97, 95, 111, 114, 89, 95, 126, 98)
	
	str(y)  sample size = 25 (t test needed)
	mean(y)  sample mean = 98.44
	sd(y)  sd of sample y = 13.09287
	median(y)  median = 98
	summary(y)
	
	plot(density(y),
	main="Distribution of class IQ",
	xlab="IQ")
	
	confidence coefficient = .90 add and subtract 
	df (degrees of freedom) for a t-test is n-1 = 25-1 = 24
	
	error <- qt(0.95, df=24)*13.09/sqrt(24) to find the standard error with 90% confidernce interval, df of 24, an sd of 13.09 and an 'n' of 25.
	print(error) error of 4.571451
	interval_1 <- 98.44-error - interval to the left side of the mean
	interval_2 <- 98.44+error - interval to the right side of the mean
	
	interval_1 93.86855
	interval_2 103.0155
	
	So the 90\% confidence interval for the average student IQ in the school is 93.87 - 103
	
	\item Next, the school counselor was curious  whether  the average student IQ in her school is higher than the average IQ score (100) among all the schools in the country.\\ 
	
	\noindent Using the same sample, conduct the appropriate hypothesis test with $\alpha=0.05$.
	
	Question1 - Part 2
	
	Null Hypothesis > Average school IQ is equal or less than 100
	Alternative Hypothesis > Average school IQ is greater than 100
	α = 0.05
	
	t.test(y, mu = 100, alternative = "greater")
	
	data:  y
	= 37.593, df = 24, p-value < 2.2e-16
	alternative hypothesis: true mean is greater than 0
	95 percent confidence interval:
	93.95993      Inf
	sample estimates:
	mean of x 
	 98.44 
	
	 Results
	 p-value < 2.2e-16 which is not less than α (0.05), therefore I fail the reject the null hypothesis, so the average school IQ is not higher than the average IQ among all the schools in the country, but less than or equal to it. 
	
\end{enumerate}

\newpage

	\section*{Question 2 (50 points): Political Economy}

\noindent Researchers are curious about what affects the amount of money communities spend on addressing homelessness. The following variables constitute our data set about social welfare expenditures in the USA. \\
\vspace{.5cm}


\begin{tabular}{r|l}
	\texttt{State} &\emph{50 states in US} \\
	\texttt{Y} & \emph{per capita expenditure on shelters/housing assistance in state}\\
	\texttt{X1} &\emph{per capita personal income in state} \\
	\texttt{X2} &  \emph{Number of residents per 100,000 that are "financially insecure" in state}\\
	\texttt{X3} &  \emph{Number of people per thousand residing in urban areas in state} \\
	\texttt{Region} &  \emph{1=Northeast, 2= North Central, 3= South, 4=West} \\
\end{tabular}

\vspace{.5cm}
\noindent Explore the \texttt{expenditure} data set and import data into \texttt{R}.
\vspace{.5cm}
\lstinputlisting[language=R, firstline=54, lastline=54]{PS01.R}  
\vspace{.5cm}
\begin{itemize}

\item
Please plot the relationships among \emph{Y}, \emph{X1}, \emph{X2}, and \emph{X3}? What are the correlations among them (you just need to describe the graph and the relationships among them)?
\vspace{.5cm}
\item
Please plot the relationship between \emph{Y} and \emph{Region}? On average, which region has the highest per capita expenditure on housing assistance?
\vspace{.5cm}
\item
Please plot the relationship between \emph{Y} and \emph{X1}? Describe this graph and the relationship. Reproduce the above graph including one more variable \emph{Region} and display different regions with different types of symbols and colors.
\end{itemize}

Answers

Problem 2

Q - Please plot the relationships among Y, X1, X2, and X3? What are the correlations among them (you just need to describe the graph and the relationships among them)?

expenditure <- read.table("https://raw.githubusercontent.com/ASDS-TCD/StatsI_Fall2021/main/datasets/expenditure.txt", header=T)
str(expenditure)  an overall view of the data set
class(expenditure)
typeof(expenditure)
ls(expenditure)
ls.str(expenditure)
summary(expenditure)
View(expenditure)

plot(density(expenditure$Y), col = "blue", main = "Expenditure on shelters/housing")
Yplot(density(expenditure$X1), col ="Red", main = "Personal income in state")
plot(density(expenditure$X2), col = "Green", main = "Financially insecure residents per 100,000")
plot(density(expenditure$X3), col = "Orange", main = "People per thousand residing in urban areas")

library(tidyverse) loads the tidyverse library

ggplot(data = expenditure) + geom_point(mapping = aes(x = Y, y = X1)) create a scatterplot using Y, X1 as variables
 main="Scatterplot Y-X1"  I hoped this would give it a title but that didn't work
xlab="per capita expenditure on shelters/housing assistance in state" I hoped this would give the x-axis a label but that didn't work
ylab="per capita personal income in state" I hoped this would give the y-axis a label but that didn't work
ggplot(data = expenditure) + geom_point(mapping = aes(x = Y, y = X2)) 
ggplot(data = expenditure) + geom_point(mapping = aes(x = Y, y = X3)) 
ggplot(data = expenditure) + geom_point(mapping = aes(x = X1, y = X2))
ggplot(data = expenditure) + geom_point(mapping = aes(x = X1, y = X3))
ggplot(data = expenditure) + geom_point(mapping = aes(x = X2, y = X3)) 

Q - Please plot the relationship between Y and Region? On average, which region has the highest per capita expenditure on housing assistance?

Region_1 <- subset(expenditure, Region == 1) makes a subset of Region 1
Region_2 <- subset(expenditure, Region == 2) makes a subset of Region 2
Region_3 <- subset(expenditure, Region == 3) makes a subset of Region 3
Region_4 <- subset(expenditure, Region == 4) makes a subset of Region 4

Region1_mean <- mean(Region_1$Y) mean per capita expenditure in Region 1
Region2_mean <- mean(Region_2$Y) mean per capita expenditure in Region 2
Region3_mean <- mean(Region_3$Y) mean per capita expenditure in Region 3
Region4_mean <- mean(Region_4$Y) mean per capita expenditure in Region 4

Region1_mean
Region2_mean
Region3_mean
Region4_mean checking the means from each region

aggregate(expenditure$Y, by = list(expenditure$Region), FUN = mean) finding the mean of variable Y by Region

Region_means <- aggregate(expenditure$Y, by = list(expenditure$Region), FUN = mean)
class(Region_means)  'Region_means is now a data.frame

Northeast <- Region_1
North_Central <- Region_2
South <- Region_3
West <- Region_4  trying to rename regions for graph, didn't work

barplot(Region_means$x,
main = "Per capita expenditure on shelters/housing assistance in state by Region", title
names.arg = Region_means$Group.1, vector with our axis names
las = 1) Rotating our axis labels)

From the plot it seems that region 4 (West) spends the most per capita on shelters / housing assistance.

Q - Please plot the relationship between Y and X1? Describe this graph and the relationship. Reproduce the above graph including one more variable Region and display different regions with different types of symbols and colors.

ggplot(data = expenditure) + geom_point(mapping = aes(x = Y, y = X1)) #relationship between Y and X1 (again)
There seems to be a positive correlation between per capita expenditure on shelters/housing assistance in state and per capita personal income in state

ggplot(data = expenditure, mapping = aes(x = Y, y = X1)) + geom_point(aes(color = Region))
Variable 'region', added with different coloured points added to the scatterplot for each region.

\end{document}
